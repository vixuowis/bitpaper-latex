
%导言段 定义纸大小 引入模版 中文支持 页边距设置 页眉页脚设置
\documentclass[a4paper]{ctexrep}
	\usepackage{xeCJK,geometry, fancyhdr,abstract}
	
	%宋体
	\setCJKmainfont{SimSun} 
	\geometry{left=2.7cm,right=2.7cm,top=3.5cm,bottom=2.5cm} %页边距
	\CTEXsetup[number={\arabic{chapter}}]{chapter}
	
	%设置首页 不加时间
	\title{基于OVAL的漏洞管理与发现系统}
	\author{夏映晖}
	\date{}
	
	%设置页眉页脚
	\pagestyle{fancy}
	\lhead{}
	\chead{北京理工大学工程硕士学位论文}
	\rhead{}
	\renewcommand{\headrulewidth}{0.5pt}
	
	%使含有Chapter页面不清除页眉页脚
	\fancypagestyle{plain}{}

%文档开始
\begin{document}
	%首页
	\maketitle
	
	%中文摘要
	\begin{abstract}
		本文……。(摘要是一篇具有独立性和完整性的短文,应概括而扼要地反映出本论文的主要内容。包括研究目的、研究方法、研究结果和结论等,特别要突出研究结果和结论。中文摘要力求语言精炼准确,硕士学位论文摘要建议500~800字,博士学位论文建议1000~1200字。摘要中不可出现参考文献、图、表、化学结构式、非公知公用的符号和术语。英文摘要与中文摘要的内容应一致。)
	
		关键词:形状记忆;聚氨酯;织物;合成;应用 (一般选3~8个单词或专业术语,且中英文关键词必须对应。)
	\end{abstract}
	
	%英文摘要
	\renewcommand\abstractname{abstract}
	\begin{abstract}
		In order to exploit …….
	
		Key Words: shape memory properties; polyurethane;textile;synthesis;application
	\end{abstract}
	
	%目录页
	\tableofcontents
	%新页 计页数
	\newpage
	\setcounter{page}{1}
	
	%第一章
	\chapter{引言}
		%第一节
		\section{背景介绍}
			This is where you will write your content. 在这里写上内容。
			
		%第二节
		\section{需求} 
			中国在East Asia. 
			%subsection
			\subsection{Hello Beijing}
				北京是capital of China. 
				%subsubsection
				\subsubsection{Hello Dongcheng District}
					%paragraph
					\paragraph{Tian'anmen Square}
					is in the center of Beijing 
						%subparagraph
						\subparagraph{Chairman Mao}
						is in the center of 天安门广场。
			%subsection 
			\subsection{Hello BIT} 
				%paragraph
				\paragraph{北京理工大学} is one of the best university in 北京。
	
	%参考文献
	\begin{thebibliography}{}
	\end{thebibliography}
\end{document}
%文档结束
