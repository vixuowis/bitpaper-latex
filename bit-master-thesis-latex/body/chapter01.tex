%%==================================================
%% chapter01.tex for BIT Master Thesis
%% based on CASthesis
%% modified by tingzhen.li@gmail.com
%% version: 0.1
%% Encoding: UTF-8
%% last update: Dec, 2012
%%==================================================

\chapter{这是什么}
\label{chap:what}

这是北京理工大学硕士学位学位论文~\LaTeX~模板,其由上海交大版本
\footnote{网页见\url{http://blog.sina.com.cn/s/blog_5e16f1770100nsqy.html}}修改得到,
修改依据《北京理工大学博士、硕士学位论文撰写规范》。

\section{模板说明}
\label{sec:fastguide}

\subsection{模板特性}
\label{sec:features}

这个模板基于~CASthesis--0.1j~文档类,中文解决方案是~\XeTeX/\LaTeX~。参考
文献建议使用~BibTeX~管理,可以生成符合国标~GBT7714~风格的参考文献列表。模
板在~Windows~和~Linux~下测试通过,更详细的系统要求请参考
\ref{sec:requirements}。

模板的外观表现和功能都放
在~bitmaster-xetex.cls~和~bitmaster-xetex.cfg~中,在对外观进行细微调整
时,只需要更新这两个文件,不需要对.tex源文件做修改。这也给模板更新带来了
极大方便。

最后,给出一个列表,罗列一下这个模板的功能要点:

\begin{itemize}
\item \inv  使用~\XeTeX~引擎处理中文,{\color{red} 我在ubuntu上安装了
  texlive2011和windows中文字体,具体方法可参见\url{http://blog.163.com/warrior511@126/blog/static/16798651220114984811658/}};
\item \inv 包含中文字符的源文件(.tex, .bib, .cfg),编码都使用UTF-8;
\item \inv 使用~BibTeX~管理参考文献。参考文献表现形式(格式)受~.bst~控制,方便
  在不同风格间切换,目前生成的列表符合国标GBT7714要求;
\item \inv 可以直接插入EPS/PDF/JPG/PNG格式的图像,并且\emph{不需要}~bounding box~文件(.bb)。
\item \inv 模板的格式受~bitmater-xetex.cls~和~bitmaster-xetex.cfg~控制,方便模
  板更新和模板修改。
\end{itemize}

\subsection{系统要求}
\label{sec:requirements}

要使用这个模板协助你完成研究生学位论文的创作,下面的条件必须满足:

\begin{itemize}
\item \inv 操作系统字体目录中有中文字体(adobe或windows字体均可);
\item \inv \TeX~系统有~\XeTeX~引擎;
\item \inv \TeX~系统有~ctex~宏包;
\item \inv \TeX~系统的fontspec宏包(fontspec.sty)足够新;
\item \inv 你有使用~\LaTeX~的经验。
\end{itemize}

 
\subsection{模板文件布局}
\label{sec:layout}

\begin{lstlisting}[basicstyle=\small\ttfamily,caption={模板文件布局},label=layout,float,numbers=none]
  ├─ diss.tex
  ├── README.pdf
  ├── bitmaster-xetex.cfg
  ├── bitmaster-xetex.cls
  ├── body
  │   ├── abstract.tex
  │   ├── app1.tex
  │   ├── app2.tex
  │   ├── chapter01.tex
  │   ├── chapter02.tex
  │   ├── conclusion.tex
  │   ├── projects.tex
  │   ├── pub.tex
  │   ├── resume.tex
  │   ├── symbol.tex
  │   └── thanks.tex
  ├── figures
  │   └── chap2
  ├── GBT7714-2005NLang.bst
  ├── reference
  │   ├── chap1.bib
  │   └── chap2.bib
  ├── run.sh
  
      
\end{lstlisting}

你拿到手的模板文件大致会包含代码\ref{layout}所列的文件,乍看起来还是挺令
人头大的。并且,这还是“干净”的时候,等到真正开始处理的时候,会冒出相当多
的“中间文件”,这又会使情况变得更糟糕。所以,有必要对这些文件做一些简要说
明。看完这部分以后,你应该发现,其实你要关心的文件类型并没有那么多。

\subsubsection{格式控制文件}
\label{sec:format}

格式控制文件控制着论文的表现形式,包括以下几个文件:
bitmaster-xetex.cfg, bitmaster-xetex.cls~和~GBT7714-2005NLang.bst。
其中,``.cfg''和``.cls''控制论文主体格式,``.bst''控制参考文献条目的格式,

一般用户最好``忽略''格式控制文件的存在,不要去碰它们。
有其他格式需要,欢迎到板上发贴。
对于因为擅自更改格式控制文件出现的问题,我一概不负责。{\large\smiley}

\subsubsection{主控文件~demo.tex}
\label{sec:demotex}

主控文件~demo.tex~的作用就是将你分散在多个文件中的内容``整合''成一篇完整的论文。
使用这个模板撰写学位论文时,你的学位论文内容和素材会被``拆散''到各个文件中:
譬如各章正文、各个附录、各章参考文献等等。
在~demo.tex~中通过``include''命令将论文的各个部分包含进来,从而形成一篇结构完成的论文。
封面页中的论文标题、作者等中英文信息,也是在~demo.tex~中填写。
部分可能会频繁修改的设置,譬如行间距、图片文件目录等,我也放在了demo.tex中。
你也可以在demo.tex中按照自己的需要引入一些的宏包
\footnote{我对宏包的态度是:只有当你需要在文档中使用那个宏包时,才需要在导言区中用~usepackage~引入该宏包。如若不然,通过usepackage引入一大堆不被用到的宏包,必然是一场灾难。由于一开始没有一致的设计目标,\LaTeX~的各宏包几乎都是独立发展起来的,因重定义命令导致的宏包冲突屡见不鲜。}。

大致而言,在~demo.tex~中,大家只要留意把``章''一级的内容,以及各章参考文
献内容包含进来就可以了。需要注意,处理文档时所有的操作命令
{}\cndash{}xelatex, bibtex等,都是作用在~demo.tex~上,而\emph{不是}后面这
些``分散''的文件,请参考\ref{sec:process}小节。

\subsubsection{论文主体文件夹body}
\label{sec:thesisbody}

这一部分是论文的主体,是以``章''为单位划分的。

正文前部分(frontmatter):中英文摘要(abstract.tex)。其他部分,诸如中英文封
面、授权信息等,都是根据~demo.tex~所填的信息``画''好了,不单独弄成文件。

正文部分(mainmatter):自然就是各章内容~chapter\emph{xxx}.tex~了,这部分无法自动生成{\LARGE\Smiley}

正文后的部分(backmatter):附录(app\emph{xx}.tex);致谢(thuanks.tex);攻读
学位论文期间发表的学术论文目录(pub.tex);个人简历(resume.tex)。参考文献列
表是``生成''的,也不作为一个单独的文件。另外,学校的硕士研究生学位论文模
板中,也没有要求加入个人建立,所以我没有在~demo.tex~中引入resume.tex。

\subsubsection{图片文件夹~figures}
\label{sec:figuresdir}

figures~文件夹放置了需要插入文档中的图片文件(PNG/JPG/PDF/EPS),建议按章再
划分子目录。

\subsubsection{参考文献数据库文件夹~reference}
\label{sec:bibdir}

reference~文件夹放置的是各章``可能''会被引用的参考文献文件。参考文献的元
数据,例如作者、文献名称、年限、出版地等,会以一定的格式记录在纯文本文
件.bib中。最终的参考文献列表是BibTeX处理.bib后得到的,名为~demo.bbl。将参
考文献按章划分的一个好处是,可以在各章后生成独立的参考文献,不过,现在看
来没有这个必要。关于参考文献的管理,可以进一步参考第\ref{chap:example}章
中的例子。

\subsection{如何使用}
\label{sec:process}

模板使用~\XeTeX~引擎提供的~xelatex~的命令处理,作用于“主控文档”demo.tex。
并且,可以省略扩展名。完整的处理流程是:
\begin{enumerate}
\item[] \inv xelatex -no-pdf -{}-interaction=nonstopmode demo
\item[] \inv bibtex demo 
\item[] \inv xelatex -no-pdf -{}-interaction=nonstopmode demo 
\item[] \inv xelatex -{}-interaction=nonstopmode demo 
\end{enumerate}

运行bibtex的时候会提示一些错误,猜测是~{{\sc Bib}\TeX}~对UTF-8支持不充
分,一般不影响最终结果。留意因为拼写错误导致的``找不到文献错误''即可。

加入~\verb|--interaction=nonstopmode|~参数是不让错误打断编译过程。
\XeTeX~仍存在一些宏包兼容性问题,所以会产生一些莫名其妙的错误(通常是重定
义错误),而这些错误通常不会影响最终的编译结果\footnote{xunicode宏包就很蛋
  痛地重定义了几个数学符号,还有诸如$\backslash$nobreakspace命令}。为方便
使用,我把处理过程写到了run.sh(for Linux)和run.bat(for Windows)批处理文件
中。更规范的应该是使用Makefile,可惜笔者功力不够,希望哪位高人把好用的
Makefile补上。

基本处理流程就是这样,一些~\LaTeX~排版的小例子可以参考第二章。

\section{硕士学位论文格式的一些说明}
\label{sec:thesisformat}


所有关于研究生学位论文模板的要求,我参考了《北京理工大学博士、硕士学位论
  文撰写规范》和研究生院提供的相关参考模板。可惜的是,很多地方参考模板同
规范相违背。例如,规范中指出,目录一级节标题采用小四、加粗,而在模板中并未
加粗。对于这种问题,基本上以大多数同学采用的方式保持一致,没有严格按照规
范进行。

这个模板是为``单面打印''准备的,但也支持``双面打印''。你可以在~demo.tex~
中设定文档类的语句中进行相应修改:

\begin{quote}
  {\small\verb+\documentclass[cs4size, a4paer, cs4size, oneside, openany]{bitmaster-xetex}+}
\end{quote}


关于页眉页脚,按照BIT要求:页眉为``北京理工大学XX学位论文'',XX表示博士或
硕士,宋体、5号,居中排列;页眉从中文摘要开始标注,论文页眉奇偶页相同。页
码从第1章(绪论)开始按阿拉伯数字(1,2,3……)连续编排,之前的部分(中文摘
要、Abstract、目录等)用大写罗马数字(Ⅰ,Ⅱ,Ⅲ……)单独编排。

研究生院要求参考文献必须符合~GBT7714~风格,学校明确提出使用这个标准而不是
自己拍脑袋想出别的做法,应该算是谢天谢地了。使用这个模板,结合BibTeX,可
以很方便地生成符合GB标准的参考文献列表。

\section{bit-master-thesis模板类简介}
\label{sec:bit.cls}
论文模板主要在bit-master-thesis.cls文件中进行定义,现对其进行简单介绍。
\subsection{页面设置}
页边距设置如下,{\color{red}好像存在问题,需进一步解决}:
\begin{lstlisting}
\usepackage[top=3.5cm,headheight=25mm,headsep=3mm,footskip=8mm,bottom=2.5cm,left=2.7cm,right=2.7cm]{geometry}
\end{lstlisting}

行距离设置,按照要求,应该为22榜,如下设置,效果基本相同:
\begin{lstlisting}
\RequirePackage{setspace}
\setstretch{1.4}
\end{lstlisting}

\subsection{章节格式与目录}
严格按照规范,采用如下代码实现:
\begin{lstlisting}
%% 设置章节格式
\CTEXsetup[number={\arabic{chapter}},name={第,章},
            nameformat={\bfseries\heiti\centering\zihao{3}},
            titleformat={\bfseries\heiti\zihao{3}},
            afterskip={30pt}]{chapter}
\CTEXsetup[nameformat={\bfseries\heiti\zihao{4}},
            titleformat={\bfseries\heiti\zihao{4}}]{section}
\CTEXsetup[nameformat={\bfseries\heiti\zihao{-4}},
            titleformat={\bfseries\heiti\zihao{-4}}]{subsection}
\CTEXsetup[nameformat={\bfseries\zihao{-4}},
            titleformat={\bfseries\zihao{-4}}]{subsubsection}
\CTEXsetup[format={\Large\bfseries}]{section}
\CTEXsetup[beforeskip={10pt}]{chapter}

%% 用\textsf{titletoc}设定目录格式。
\RequirePackage{titletoc}
\titlecontents{chapter}[0pt]{\vspace{0.25\baselineskip} \songti \zihao{4}}
    {\thecontentslabel\quad}{}
    {\hspace{.5em}\titlerule*{.}\contentspage}
\titlecontents{section}[2em]{\songti \zihao{-4}}
    {\thecontentslabel\quad}{}
    {\hspace{.5em}\titlerule*{.}\contentspage}
\titlecontents{subsection}[4em]{\songti \zihao{-4}}
    {\thecontentslabel\quad}{}
    {\hspace{.5em}\titlerule*{.}\contentspage}
\end{lstlisting}

\subsection{封面设计}
这里我们为封面设计提供了众多命令,以中文封面为例:
\begin{lstlisting}
%%%%中文标题页的可用命令
\newcommand\classification[1]{\def\CAST@value@classification{#1}}
\newcommand\studentnumber[1]{\def\CAST@value@studentnumber{#1}}
\newcommand\confidential[1]{\def\CAST@value@confidential{#1}}
\newcommand\UDC[1]{\def\CAST@value@UDC{#1}}
\newcommand\serialnumber[1]{\def\CAST@value@serialnumber{#1}}
\newcommand\school[1]{\def\CAST@value@school{#1}}
\newcommand\degree[1]{\def\CAST@value@degree{#1}}
\renewcommand\title[2][\CAST@value@title]{%
  \def\CAST@value@title{#2}
  \def\CAST@value@titlemark{\MakeUppercase{#1}}}
\renewcommand\author[1]{\def\CAST@value@author{#1}}
\newcommand\advisor[1]{\def\CAST@value@advisor{#1}}
\newcommand\advisorinstitute[1]{\def\CAST@value@advisorinstitute{#1}}
\newcommand\major[1]{\def\CAST@value@major{#1}}
\newcommand\submitdate[1]{\def\CAST@value@submitdate{#1}}
\newcommand\defenddate[1]{\def\CAST@value@defenddate{#1}}
\newcommand\institute[1]{\def\CAST@value@institute{#1}}
\newcommand\chairman[1]{\def\CAST@value@chairman{#1}}
\end{lstlisting}

使用这些命令,即可在主控文件中设置自己的封面,例如本文档在demo.tex中如下
设置:
\begin{lstlisting}
\classification{TQ028.1}
\UDC{111}
\title{\LARGE{BIT硕士论文\LaTeX 模板}}
\author{大眼小蚂蚁}
\institute{机电学院}
\advisor{教授}
\chairman{离散}
\degree{工学硕士}
\major{兵器科学与技术}
\school{北京理工大学}
\defenddate{2012年12月}
\studentnumber{2120100277}
\end{lstlisting}

这些变量设置好之后,\\使用\verb+\maketitle+产生封面的第一二页;\\使用
\verb+\makeenglishtitle+产生英文标题页;\\使用\verb+\makeVerticalTitle+产
生竖着排放的标题页;\\使用\verb+\makeDeclareOriginal+产生声明页。

